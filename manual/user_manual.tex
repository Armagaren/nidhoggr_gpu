\documentclass[12pt]{article}
\usepackage{geometry}
\usepackage{hyperref}
\usepackage{graphicx}
\usepackage{courier}
\geometry{margin=1in}

\title{Nidhoggr User Manual}
\author{Cody Raskin}
\date{\today}

\begin{document}
	
	\maketitle
	
	\newpage
	
	\begin{center}
		\begin{verbatim}
     _   _ _
 ___|_|_| | |_ ___ ___ ___ ___
|   | | . |   | . | . | . |  _|
|_|_|_|___|_|_|___|_  |_  |_|
v0.8.0            |___|___|
		\end{verbatim}
	\end{center}
	
	\newpage
	
	\tableofcontents
	\newpage
	
	\section{Introduction}
	\subsection{Purpose of Nidhoggr}

Nidhoggr is a generic physics simulation framework. It is designed to be used as a base for varied physics simulation methods (FVM,FEM,etc) while keeping helper methods likes equations of state and integrators generic enough to be portable to a wide variety of methods choices. Nidhoggr's major classes and methods are written in C++ and wrapped in Python using pybind11 to enable them to be imported as Python3+ modules inside a runscript. Python holds and passes the pointers to most objects inside the code, while the integration step is always handled by compiled C++ code. Any Python class that returns the expected data types of the compiled C++ classes can substitute for a precompiled package (e.g. a custom equation of state), though speed will suffer. 

	\subsection{Overview of capabilities}
	
As of \date{\today}, Nidhoggr's capabilities are as follows:
\begin{table}[h!]
	\centering
	\begin{tabular}{|l|l|l|l|}
		\hline
		\textbf{Component} & \textbf{Working} & \textbf{Development} & \textbf{Planned} \\
		\hline
		Physics & N-body gravity & HLL hydro & SPH\\ 
		& Gravity point sources & FEM &\\
		& Constant direction gravity sources &  &\\
		& Particle kinetics &  &\\
		& Acoustic wave solvers &  &\\
		& Shallow wave equation solvers &  &\\
		& Chemical reaction solvers &  &\\
		\hline
		Equations of State & Ideal gas &  & Helmholtz\\ 
		& Polytrope &  &\\
		\hline
		Time Integrators & Forward Euler &  & Symplectic\\ 
		& 2nd Order Runge-Kutta &  &\\
		& 4th Order Runge-Kutta &  &\\
		\hline
		Meshing & Eulerian grid & FEM & AMR\\
		\hline
		Data IO & Silo &  &\\
		& vtk &  &\\
		& obj &  &\\
		& wav &  &\\
		\hline
		Custom Data Types & Vectors & Elements &\\
		& Tensors &  &\\
		& Cosmologies & & \\
		& Units & & \\
		\hline
		Parallel & OpenMP & & MPI \\
		\hline
	\end{tabular}
	\caption{Status of Major Components in Nidhoggr}
	\label{tab:component-status}
\end{table}
	
	\subsection{Intended audience}
	
Nidhoggr's intended audience is computational scientists who want a toy simulation code to scope simple problems with that's easily driveable and scriptable with a Python interface, and anyone who doesn't mind getting their hands dirty writing their own physics packages in a fully abstracted simulation framework. 

\newpage
	
	\section{Installation}
	\begin{itemize}
		\item System requirements
		\item Dependencies
		\item Downloading the source code
		\item Building and installing
	\end{itemize}

\newpage
	
	\section{Getting Started}
	\begin{itemize}
		\item Basic concepts
		\item First run: a simple example
	\end{itemize}

\newpage
	
	\section{Usage}
	\begin{itemize}
		\item Running Nidhoggr
		\item Command-line options
	\end{itemize}

\newpage
	
	\section{Core Concepts}
	\begin{itemize}
		\item Nodelists and Fields
		\item Physics methods
		\item Equations of State
		\item Mesh/grid handling
		\item Boundary conditions
		\item Integrators
		\item The Controller
	\end{itemize}

\newpage
	
	\section{Examples}
	
The \texttt{examples} folder holds Python runscripts that solve a particular notional physics (or purely calculational) problem. 

\begin{table}[h!]
	\centering
	\begin{tabular}{|l|l|}
		\hline
		\textbf{File} & \textbf{Purpose}\\
		\hline
		\texttt{cherenkov.py} & Simulates a supersonic (or superluminal) point source \\
		& moving through a medium at a speed greater than c. \\
		\hline		
		\texttt{cosmo.py} & Creates an example cosmology ($\Omega_m$,$\Lambda$,$H_0$) and reports \\
		& the properties of that cosmology at the chosen redshift. \\
		\hline
		\texttt{diffractionGrating.py} & Simulates the transmission of an acoustic wave through \\
		& a diffraction grating.\\
		\hline
		\texttt{imageToStringArt.py} & Creates the instructions for (and previews) an image made \\
		& from strings stretched across a wheel with a chosen number \\
		& of pins. \\
		\hline
		\texttt{oort.py} & Simulates a star passing through the Oort cloud \\
		& and dislodging a comet from its orbit. \\
		\hline
		\texttt{plinko.py} & Simulates the Plinko game. \\
		\hline
		\texttt{relativity.py} & Calculates the time dilation for a relativistic traveler. \\
		\hline
		\texttt{rps.py} & Simulates the destruction of chemical mixtures in a \\
		& rock-paper-scissors-like reaction setup, where \\
		&  $A\to B\to C\to A$.\\
		\hline
		\texttt{tensors.py} & Creates some tensors and does some linear algebra with them.\\
		\hline
		\texttt{vectors.py} & Creates some vectors and does some linear algebra with them.\\
		\hline
		\texttt{waveLogo.py} & Simulates acoustic waves inside a region with Dirichlet\\
		& boundary conditions arranged in a unique fashion.\\
		\hline
	\end{tabular}
	\caption{Examples included in the main branch.}
	\label{tab:component-status}
\end{table}

	\subsection{Simple test cases}

Many of the Python scripts inside the \texttt{tests} folder stress single components of Nidhoggr, or a small subset of them. For instance, \texttt{waveBox.py} tests the acoustic wave solver with a single oscillatory source in the center of a box with two openings on either end (using Dirichlet boundaries to create the box). 

\newpage
	
	\section{Customization and Extension}
	\begin{itemize}
		\item Modifying source code
		\item Adding new physics modules
		\item Extending the input parser
	\end{itemize}

\newpage
	
	\section{Best Practices}
	\begin{itemize}
		\item Tips for efficient simulation
		\item Debugging guidance
		\item Performance tuning
	\end{itemize}

\newpage
	
	\section{Troubleshooting}
	\begin{itemize}
		\item Common errors and solutions
		\item FAQ
	\end{itemize}

\newpage
	
	\section{Reference}
	\begin{itemize}
		\item Code structure overview
		\item Important classes and functions
		\item File organization
	\end{itemize}

\newpage
	
	\section{Acknowledgments}
	\begin{itemize}
		\item Contributors
		\item Funding and support
	\end{itemize}
	
	\section{License}
	\begin{itemize}
		\item License terms
		\item How to cite Nidhoggr
	\end{itemize}

\newpage
	
	\appendix
	
	\section{Appendix A: Glossary}
	\begin{itemize}
		\item Terms and definitions
	\end{itemize}
	
	\section{Appendix B: Additional Resources}
	\begin{itemize}
		\item Related software
		\item Recommended reading
	\end{itemize}
	
\end{document}
